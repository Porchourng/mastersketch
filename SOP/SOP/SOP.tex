%% Based on a TeXnicCenter-Template by Gyorgy SZEIDL.
%%%%%%%%%%%%%%%%%%%%%%%%%%%%%%%%%%%%%%%%%%%%%%%%%%%%%%%%%%%%%
 %
\documentclass[a4paper,12pt]{article}%
 
 \usepackage{graphicx}
%
 
\usepackage[latin2]{inputenc}
%
%---- Setting the margins
%
\setlength{\textwidth}{160mm}%
\setlength{\hoffset}{-0.4mm}%
\setlength{\oddsidemargin}{0mm}%
\setlength{\topskip}{0mm}%
\setlength{\textheight}{240mm}%
\setlength{\topmargin}{0mm}%
\setlength{\voffset}{-5.4mm}
%
\linespread{2}
\begin{document}
\title{Statement of Purpose}
\author{Maha El Meseery}
 
\date{\today}
 
\maketitle 

%-------------------------------------------------------------
%
 %-------------------------------------------------------------

%\begin{letter}%
\section{Outline Or What to write}
You have to really dig. Be introspective. Don't settle for I love this field. Why do you love this field? Why do you want to work in this field for the rest of your life? Why does it complete you? 
\subsection{General Notes}
\label{sec:General}
  \begin{enumerate}
	\item  If, for instance, you talk about your deep desire to make society a better place, your application should reflect it. Have you done anything about this desire? Can you talk about your actions and experiences\\
	  \hrulefill
	  
	  \hrulefill
	  
	  	  \hrulefill
	  	  
	  \hrulefill
\item How to make the essay unique 
 One of the best ways to do this is to discuss, briefly, an idea in your field that turns you on intellectually. It's an effective essay-opener, and it lets you write about something besides yourself for a bit. There are other benefits as well. The idea you choose to talk about, and your comments on it, often tell an admissions committee more about you than your own self-descriptions can.
Discussing an idea will catch people's attention and give your readers a reprieve from people writing about themselves. That me-me-me  stuff can get irritating after a while, even though it's what the prompt asks for. A discussion of an idea demonstrates your interest in your field, rather than just describing it. Ultimately, that's more convincing.\\

	  \hrulefill
	  
	  \hrulefill
\item do not use love reading,......What do you love about it? How does it affect you? What do you want others to know about how and why you read? Who do you most hope to reach through your reading and writing?\\

	  \hrulefill
	  
	  \hrulefill
	  
\item Personal stories can sometimes be effective, particularly stories of hardships overcome or of an emerging sense of purpose. 
\item  	What makes me different from other grad school applicants?
\item What is important to you? Why is it important? How did you develop these values? In particular, which  experiences and activities influenced your values and reflect them?
\item What is distinctive about you? What anecdotes reveal that distinctiveness?\\

	  \hrulefill
	  
	  \hrulefill
	  
\item What are you proud of and why?
\item 	Be specific, not general.
\item Show, dont tell (Use examples to demonstrate your abilities)
\begin{itemize}
	\item Hobbies 
	\item Projects that you've completed
	\item Jobs
	\item Responsibilities
	\item Accomplishments in the personal and scholastic arena
	\item Major life events that have changed you
	\item Challenges and hurdles you've overcome
	\item Life events that motivate your education
	\item People who have influenced you or motivated you 
	\item Traits, work habits, and attitudes that will insure your success your goals
\end{itemize}

	  \hrulefill
	  
	  \hrulefill

\item   Here's an organization I would recommend: (1) passionate hook; (2) segu� to your background in the field; (3) specific classes by title and professors you have had (especially if well-known in the field); (4) related extracurricular activities (especially if they hint at some personal quality you want to convey); (5) any publications or other professional accomplishments in the field (perhaps conference presentations or public readings); (6) explanations about problems in your background (if needed); and (7) why you have chosen this grad school (name one or two professors and what you know of their specific areas or some feature of the program which specifically attracts you).
	\end{enumerate}		
	  \hrulefill
	  
	  \hrulefill 
	  
	  \hrulefill
\subsection{Introductions}
\label{sec:Introductions}
\begin{enumerate}
	\item	A summary of your accomplishments first	
	\item 	Your own appraisal of yourself (strengths, weaknesses, uniqueness)
		\begin{enumerate}
			\item 	Leadership role activities
			\item 	Career objectives and goals	
		\end{enumerate}
	  \hrulefill
	  \hrulefill
\item Personal stories can sometimes be effective, particularly stories of hardships overcome or of an emerging sense of purpose. \\
\hrulefill

	  \hrulefill
	  
\item  	What makes me different from other grad school applicants?
	\item  	Personal and other areas of interest (Hobbies, sports, social or leisure).
	\item  	Any other information, which you feel, will support your application
	\item Give examples of personal attributes or qualities that would help you complete graduate study successfully. \\
		  \hrulefill
		  
	  \hrulefill
	  
\item  Who am I?		
\begin{itemize}
	\item What characteristics do I possess (e.g. honest, compassionate, loyal)?
	\item What skills do I have (e.g. analytical, communication, organizational)?
	\item How have I changed/grown over the years? What caused these changes and how have they affected me?
	\item What makes me unique? How am I different from other applicants?
	\item Why should the admissions committee be interested in me?
	\item Are there any obstacles that I had to overcome and how have I dealt with these difficulties from my past?
	\item Are there any experiences from my past that have affected my life? Can I relate these experiences to my goals?
	\item Who has influenced me over the years (e.g. parent, sibling, teacher, or friend) and how have they influenced me?
	\item What are my career goals?
	\item Why do I want to continue my studies?
\end{itemize}
 	  \hrulefill
 	  
		  \hrulefill
		  
	\end{enumerate}
	  \hrulefill
	  
\subsection{Motivations}
\label{sec:Motivations}
\begin{enumerate}
	\item Explain your motivations.  (of how you decided that teaching is the ideal career path for you) \\
		\item The experiences that made you want to enter this field
		\item Steps you have already taken towards these goals   
		\item why you feel so passionate about your subject area?  (1.	How did you become interested in this field)  \\
			  \hrulefill
			  
	  \hrulefill
	  
		\item	What experiences confirmed that this is what you really want to study?
	\item	When did you realize that this wasn't just a casual interest, but what you actually wanted to do with your life?
			\item	A summary of your accomplishments first
				\item Background information--people and events that influenced your decision
 	\item 		The reason why you wish to study your chosen subject\\
 		  \hrulefill
 		  
	  \hrulefill
	  
 	\item Show, dont tell (Use examples to demonstrate your abilities)
\begin{itemize}
	\item Major life events that have changed you
	\item Life events that motivate your education
	\item People who have influenced you or motivated you 
	\item Traits, work habits, and attitudes that will insure your success your goals
\end{itemize}
 		  \hrulefill	
 		   
	  \hrulefill
	  
\item  Who am I?	
\begin{itemize}
	\item What characteristics do I possess (e.g. honest, compassionate, loyal)?
	\item What skills do I have (e.g. analytical, communication, organizational)?
	\item How have I changed/grown over the years? What caused these changes and how have they affected me?
	\item What makes me unique? How am I different from other applicants?
	\item Why should the admissions committee be interested in me?
	\item Are there any obstacles that I had to overcome and how have I dealt with these difficulties from my past?
	\item Are there any experiences from my past that have affected my life? Can I relate these experiences to my goals?
	\item Who has influenced me over the years (e.g. parent, sibling, teacher, or friend) and how have they influenced me?
	\item What are my career goals?
	\item Why do I want to continue my studies?
\end{itemize}
 	  \hrulefill	 
 	  
	  \hrulefill
	  
\item When did I become fascinated by my field of study?
\begin{enumerate}
	\item Why am I interested in my field of study?
	\item What have I learned about my subject of interest?
	\item How has my discipline shaped me? What has my field of study taught me about myself?
	\item How can I address my academic record
\end{enumerate}
 		  \hrulefill	
 		   
	  \hrulefill
	  
	\end{enumerate}
	 \hrulefill
	 
	 \hrulefill
	 
\subsection{Goals}
\label{sec:Goals}
	\begin{enumerate}		
\item 	Your own appraisal of yourself (strengths, weaknesses, uniqueness)
		\begin{enumerate}		
	\item 	Leadership role activities
	\item 	  Career objectives and goals
	\end{enumerate}
		\item 		 	Your ambitions / goals / expectations.
	
\end{enumerate}
	  \hrulefill
	  
	  \hrulefill
	  
	  \hrulefill
	  
	  \hrulefill
	  
	  \hrulefill
	  
\subsection{Educations}
\label{sec:Educations}
\begin{enumerate}
 	\item	Learning experience(s) that serve as a foundation for your choice of career
\item 	Describe background characteristics that may have placed you at an educational disadvantage (English language learner, family economic history, lack of educational opportunity, disability, etc.).
	\item Leave the reader believing that you are prepared for advanced academic work and will be successful in graduate school.\\
\hrulefill

\hrulefill

	\item Do I have any gaps or inconsistencies on my records (transcript and/or exam scores) that I can explain?
 
 
 \begin{enumerate}
 
	\item Are there any awards, recognition, or honors that I have received and that are worth mentioning?
	\item How do field experiences enhance my application?
\end{enumerate}
 
	
 	\end{enumerate}
 		  \hrulefill
 		  
	  \hrulefill
	  
	  \hrulefill
	  
 	
\subsection{Experience}
\label{sec:Experience}

  \begin{enumerate}
	\item  Tailor your past experiences to your goals  (2.	
	\item What experiences have contributed toward your preparation for further study in this field?)  
\item What experiences have contributed toward your preparation for further study in this field
			\item Steps you have already taken towards these goals   
\item	Explain not only what you know about your field, but also what you don't know-where is your knowledge particularly strong 
\item what areas do you still need to learn more about in order to reach your goals?\\
	  \hrulefill
	  
	  \hrulefill
	  
	\item  Tailor your graduate school experience to your goals. (3.	What are your future goals?  		\item	A summary of your accomplishments first
					\item		Where you picked up first-hand information/experience about the field
	\item 		 	Any experience you have to past study related to your chosen subject.
	\item 		 	Any employment experience.
	\item Describe your determination to achieve your goals, your initiative and ability to develop ideas, and your ability to work independently.\\
		  \hrulefill
		  
	  \hrulefill
	  
	\item Show, dont tell (Use examples to demonstrate your abilities)
\begin{itemize}
	\item Hobbies 
	\item Projects that you've completed
	\item Jobs
	\item Responsibilities
	\item Accomplishments in the personal and scholastic arena
	\item Major life events that have changed you
	\item Challenges and hurdles you've overcome
 	\item Traits, work habits, and attitudes that will insure your success your goals
\end{itemize}
	  \hrulefill
	  
	  \hrulefill
	  
\item What internships and/or jobs have I had in the past?
	 \begin{enumerate}
	\item What have I learned from my internship and/or job experiences? What skills have I acquired through my internship and/or job?
	\item 	What have I learned from my internship and/or job experiences? What skills have I acquired through my internship and/or job?
	\item How are my internship and/or job experiences related to my field of interest? Have my internship and/or job experiences prepared me for my future career?
	\item Have I been involved in any social services? How has the experience contributed to my growth and how is it related to my goals?
	\item What extracurricular activities have I participated in and how do they contribute to my professional goals?\\
		  \hrulefill
		  
	  \hrulefill
	  
\end{enumerate}
\end{enumerate}
	  \hrulefill
	  
	  \hrulefill
	  
	  \hrulefill
	  
\subsection{Research Interest}
\label{sec:ResearchInterest}

  \begin{enumerate}
	\item What are your research interests? 
	\item  One of the best ways to do this is to discuss, briefly, an idea in your field that turns you on intellectually. It's an effective essay-opener, and it lets you write about something besides yourself for a bit. There are other benefits as well. The idea you choose to talk about, and your comments on it, often tell an admissions committee more about you than your own self-descriptions can.
Discussing an idea will catch people's attention and give your readers a reprieve from people writing about themselves. That me-me-me  stuff can get irritating after a while, even though it's what the prompt asks for. A discussion of an idea demonstrates your interest in your field, rather than just describing it. Ultimately, that's more convincing.
	\end{enumerate}
		  \hrulefill
		  
	  \hrulefill
	  
	  \hrulefill
	  
\subsection{Program and institute}
\label{sec:Program}
  \begin{enumerate}
	\item  How are you a  match  for the program to which you are applying? Explain what attracts you most to the institution/program 
	\item 		 	The reasons you wish to study in the UK.
	\item  	What makes me different from other grad school applicants?
\item  	What can I contribute to the graduate program?\\
	  \hrulefill
	  
	  \hrulefill
	  
\item Who will be reading my personal statement?
 \begin{enumerate}
 	\item How can I make my essay compelling to the readers?
	\item Why am I applying to this program?
	\item Why am I applying to this institution?
	\item How will attending this graduate school help me grow as an individual and prepare me for my future career?
	\item What do I offer the graduate program. Why should a faculty member take me on as a mentee?
	\item Considering these issues will help you find a theme for your essay. The next step entails weighing the information to decide what to include.
\end{enumerate}
	  \hrulefill
	  
	  \hrulefill
	  
\item 	 Finally, construct story that tells readers about who you are and your goals - and how grad school fits.
	\end{enumerate}
	  \hrulefill
	  
	  \hrulefill
	  
	  \hrulefill
	  
%	\item Explain your motivations.  (of how you decided that teaching is the ideal career path for you) \\
%	The experiences that made you want to enter this field\\
%	Steps you have already taken towards these goals   \\
%	why you feel so passionate about your subject area?  (1.	How did you become interested in this field)  \\
%		What experiences confirmed that this is what you really want to study?
%%	When did you realize that this wasn't just a casual interest, but what you actually wanted to do with your life?
%	\item Explain your Goals 
%	
%	\item Tailor your past experiences to your goals  (2.	What experiences have contributed toward your preparation for further study in this field?)\\  
%What experiences have contributed toward your preparation for further study in this field
%	
%	Explain not only what you know about your field, but also what you don't know-where is your knowledge particularly strong, and what areas do you still need to learn more about in order to reach your goals? 
%\subsection{ Notes:  }
\section {Organization}
\begin{enumerate}
	\item Brief Introduction  or (1) passionate hook;
	\item  Education Background  or segu� to your background in the field; 
 \item (3) specific classes by title and professors you have had (especially if well-known in the field);
	\item Employment History  or  related extracurricular activities (especially if they hint at some personal quality you want to convey);
 \item any publications or other professional accomplishments in the field (perhaps conference presentations or public readings); 
	\item Highlight Purpose for Proposed Course of Study  and explanations about problems in your background (if needed); and 
	\item Career Goals 
	\item why you have chosen this grad school (name one or two professors and what you know of their specific areas or some feature of the program which specifically attracts you).
	\item In Conclusion
\end{enumerate}
	  \hrulefill
	  
	  \hrulefill
	  
	  \hrulefill
	  
	 	  \hrulefill
	 	  
	  \hrulefill
	  
	  \hrulefill
	  
\section{Essay}
Artificial Intelligence and Image processing was two of the most interesting courses that I took in my last year in undergraduate school. 
I always thought of a computer as a super intelligent machine, understanding that simple trivial tasks for humans are considered the most challenging problems in computer science was inspiring. The knowledge was gained when I took Artificial Intelligence and Image processing in my last year in undergraduate school. Taking these courses affected my choice for graduation project, it made me think more of a research project that tends to AI and Machine Vision rather than other disciplines. My graduation project "Face Recognition" was chosen to help me pursuit more understanding on the problems of Machine Vision. 
//further increase my  in  improve my knowledge and  to I choose my graduation project  to solve one of the Machine vision problems. 
\\

 Teaching and work in resala with the orphans helped me to tone my temper and given to improve my weakness as I wanted to present an ideal for the children that they can imetate. It helped me focus on how could I improve my self more to help myself and my comunity. 
 
 \\
 My research intereset are in field of software engineering in how to use old experiences, projects, bugs, programs to help improve the quality and speed of the software engineering process. using AI, data mining methods is one of the major ideas that are used to help use old requriments in the new software project.  It may not be a silver bullet that solve the software engineering problems but adding new thinking that may help practiceners.  \\
 My interset a
 
 

 %\end{letter}
\end{document}
