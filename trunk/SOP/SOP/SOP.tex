%% Based on a TeXnicCenter-Template by Gyorgy SZEIDL.
%%%%%%%%%%%%%%%%%%%%%%%%%%%%%%%%%%%%%%%%%%%%%%%%%%%%%%%%%%%%%
 %
\documentclass[10pt]{article}%

 \usepackage{graphicx}
%
 \pdfpagewidth 8.5in
\pdfpageheight 11in

\setlength\topmargin{0in}
\setlength\headheight{0in}
\setlength\headsep{-0.5in}
\setlength\textheight{10.0in}
\setlength\textwidth{7.5in}
\setlength\oddsidemargin{-0.5in}
\setlength\evensidemargin{-0.5in}
\setlength\parindent{0.25in}
\setlength\parskip{0.25in}

\title{Statement of Purpose}
\author{Maha El Meseery }

\date{\today}
%\maketitle
\linespread{1}
\begin{document}
\large
 \begin{center}
\textbf{ Statement of  Interest \\ Maha El Meseery } \hrule
\end{center}
\normalsize

I dream of participating in building a world-class research in my home country; Egypt. I am confident that both my experiences in the academic sector (as a Research Assistant) and industrial field (as a Software Engineer) provided me with the necessary background to pursue my dream. Throughout my work as a research assistant I have published various papers in different research areas including swarm intelligence, face recognition and handwriting recognition. I also have hand-on experiences in software development in different commercial sectors. Applying to a PhD program in your esteemed institute is a step towards achieving my goal. I have an immense interest in the computer vision research conducted in your university. I believe that University of Ottawa is the most suitable place for me to pursue my PhD.

 My interest in research started early through my undergraduate study which impelled me to look for a research problem for my graduation project. I chose my project to be ``Face recognition and Identification'' where I investigated various face recognition and skin detection methods. This project showed me what it meant to work in research and how to handle the frustration of wrong or unexplained result. On the other hand, the thrill and the continuous motivation to gain more information and ideas that helps in investigating more creative problem solving techniques. In short, I knew then that I want to continue to work in a scientific research career.


 After realizing my dream I strived to work as a researcher in an academic institute. Therefore, since 2007, I have been affiliated with the 'Signal and Image Processing' research group in the Electronic Research Institute (ERI), Egypt. The position offered a research environment which I believe is essential for conducting any constructive research. My work in ERI focused on investigating and evaluating the use of swarm intelligence in image processing field. I also worked on integrating this work with my Master's thesis in Sketch Recognition and Understanding. In the master thesis, I worked on evaluating the efficiency of different Particle Swarm Optimization (PSO) algorithms on segmenting users hand-drawn sketches. ERI enabled me to increase my knowledge as I focused on joining several research projects investigating various problems including swarm intelligence in face recognition, land mine detection and data clustering. The environment helped me to engage in research discussions and projects that opened my mind to other fields of research which provide me with a breadth of information that will definitely help me as a researcher in the long term.

 	Although my dream was to become a scientific researcher, I believed that I need some practical experience. Thus, I worked for four years as a Software Engineer in EngNet Company, Egypt. During my work I was engaged in various discussions and negotiations that taught me how to resolve conflicts and compromise to reach a final decision. After the first year, I was promoted to Team Leader and Senior Software Engineer where I took charge of improving the software development process, helping developers for better collaborative work, and looking for methods that will improve issue and bug tracking systems. Since then, Software Engineering has become my passion; learning new skills on how to supervise and evaluate operations of software projects led me to search about how to implement software practices that will enhance the company performance and products.

I believe that any academic or scientific researcher must acquire some tutoring skills and experience. Thus, prior to my work in EngNet Company, I worked as Teaching Assistant in the October University for Modern Science and Art (MSA), Egypt. Additionally, I have been working as a teaching and research assistant in the American University in Cairo (AUC), Egypt since Sep. 2008. My responsibilities include preparing tutorials and course materials as well as correcting assignment and projects. I have been assisting in presenting and advising for various courses ranging from Software Engineering and AI to Logic Design and Embedded Systems. Even though my main goal is to become a researcher, teaching is in itself an enriching experience which improved my skills in both professional and personal aspects.

 
 	Besides working as a teaching assistant in AUC, I also work as a research assistant in Arabic Handwriting Recognition group directed by Dr. Sherif Abdelazeem. During my work, we proposed new domain specific features to improve recognition of Arabic Digit Recognition systems. We collected a new Arabic on-line Digits Handwriting datasets to be used as benchmarks. Additionally, I presented an efficient online Arabic Digits system which outperforms current recognition systems.  Lately, we are conducting research on efficient systems for recognizing online Arabic handwriting on mobile and embedded devices. 


 
 Image processing is my main field of interest, more specifically Pattern Recognition problems. I am interested in developing theoretical and practical solutions for medical imaging problems using pattern recognition methods. My interests emerge from the increase need for solutions that can improve current medical diagnosis problems.  Furthermore, I have an immense interest in object tracking and recognition systems as in surveillance and security systems in unconstrained environment. My other interests lay in improving gesture recognition problems leading to natural Human Computer Interaction (HCI) but I am open to any idea in the field of Pattern Recognition and HCI. 

 
 
  	In conclusion, pursuing a PhD degree will help me achieve my goal as a researcher. My choice of studying abroad is based not only on the research experience that I will gain in  studying in another university but also on the experiences I am sure I will gain by living in a diverse community.  University of Ottawa is one of the top universities in Canada which will provide me with high caliber qualities that I need to widen my vision. I am proud that my professors and supervisors see me as a responsible, reliable and self motivated person. I am confident that my experience will enable me to contribute to the on-going research at University of Ottawa.


\noindent Thank you for your time and consideration of my candidacy.




\end{document}
