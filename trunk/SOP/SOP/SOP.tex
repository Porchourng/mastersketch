%% Based on a TeXnicCenter-Template by Gyorgy SZEIDL.
%%%%%%%%%%%%%%%%%%%%%%%%%%%%%%%%%%%%%%%%%%%%%%%%%%%%%%%%%%%%%
 %
\documentclass[10pt]{article}%
 
 \usepackage{graphicx}
%
 \pdfpagewidth 8.5in
\pdfpageheight 11in

\setlength\topmargin{0in}
\setlength\headheight{0in}
\setlength\headsep{-0.5in}
\setlength\textheight{10.0in}
\setlength\textwidth{7.5in}
\setlength\oddsidemargin{-0.5in}
\setlength\evensidemargin{-0.5in}
\setlength\parindent{0.25in}
\setlength\parskip{0.25in} 
%\usepackage[latin2]{inputenc}
%
%---- Setting the margins
%
% \setlength{\textwidth}{190mm}%
% \setlength{\hoffset}{0mm}%
% \setlength{\oddsidemargin}{0mm}%
% \setlength{\topskip}{0mm}%
% \setlength{\textheight}{260mm}%
% \setlength{\topmargin}{0mm}%
% \setlength{\voffset}{0mm}
% \setlength{\oddsidemargin}{-0.5in}
% \setlength{\evensidemargin}{0in}
% \setlength{\parindent}{0.25in}
% \setlength{\parskip}{0.25in}


%
\title{Statement of Purpose}
\author{Maha El Meseery }
 
\date{\today}
%\maketitle 
\linespread{1}
\begin{document}
\large
 \begin{center}
\textbf{ Statement of Purpose \\ Maha El Meseery } \hrule
\end{center}
\normalsize
 
 
% % -------------------------------------------------------------
% %
%  %-------------------------------------------------------------
% 
% %\begin{letter}%
% \section{Outline Or What to write}
% You have to really dig. Be introspective. Don't settle for I love this field. Why do you love this field? Why do you want to work in this field for the rest of your life? Why does it complete you?
%  
% \subsection{General Notes}
% \label{sec:General}
%   \begin{enumerate}
% 	\item  If, for instance, you talk about your deep desire to make society a better place, your application should reflect it. Have you done anything about this desire? Can you talk about your actions and experiences
%  
% 	  \hrulefill
% 	  
% 	  \hrulefill
% 	  
% 	  	  \hrulefill
% 	  	  
% 	  \hrulefill
% \item How to make the essay unique 
%  One of the best ways to do this is to discuss, briefly, an idea in your field that turns you on intellectually. It's an effective essay-opener, and it lets you write about something besides yourself for a bit. There are other benefits as well. The idea you choose to talk about, and your comments on it, often tell an admissions committee more about you than your own self-descriptions can.
% Discussing an idea will catch people's attention and give your readers a reprieve from people writing about themselves. That me-me-me  stuff can get irritating after a while, even though it's what the prompt asks for. A discussion of an idea demonstrates your interest in your field, rather than just describing it. Ultimately, that's more convincing.
% 
% 	  \hrulefill
% 	  
% 	  \hrulefill
% \item do not use love reading,......What do you love about it? How does it affect you? What do you want others to know about how and why you read? Who do you most hope to reach through your reading and writing?
% 
% 	  \hrulefill
% 	  
% 	  \hrulefill
% 	  
% \item Personal stories can sometimes be effective, particularly stories of hardships overcome or of an emerging sense of purpose. 
% \item  	What makes me different from other grade school applicants?
% \item What is important to you? Why is it important? How did you develop these values? In particular, which  experiences and activities influenced your values and reflect them?
% \item What is distinctive about you? What anecdotes reveal that distinctiveness?
% 
% 
% 	  \hrulefill
% 	  
% 	  \hrulefill
% 	  
% \item What are you proud of and why?
% \item 	Be specific, not general.
% \item Show, don't tell (Use examples to demonstrate your abilities)
% \begin{itemize}
% 	\item Hobbies 
% 	\item Projects that you've completed
% 	\item Jobs
% 	\item Responsibilities
% 	\item Accomplishments in the personal and scholastic arena
% 	\item Major life events that have changed you
% 	\item Challenges and hurdles you've overcome
% 	\item Life events that motivate your education
% 	\item People who have influenced you or motivated you 
% 	\item Traits, work habits, and attitudes that will insure your success your goals
% \end{itemize}
% 
% 	  \hrulefill
% 	  
% 	  \hrulefill
% 
% \item   Here's an organization I would recommend: (1) passionate hook; (2) segue to your background in the field; (3) specific classes by title and professors you have had (especially if well-known in the field); (4) related extracurricular activities (especially if they hint at some personal quality you want to convey); (5) any publications or other professional accomplishments in the field (perhaps conference presentations or public readings); (6) explanations about problems in your background (if needed); and (7) why you have chosen this grad school (name one or two professors and what you know of their specific areas or some feature of the program which specifically attracts you).
% 	\end{enumerate}		
% 	  \hrulefill
% 	  
% 	  \hrulefill 
% 	  
% 	  \hrulefill
% \subsection{Introductions}
% \label{sec:Introductions}
% \begin{enumerate}
% 	\item	A summary of your accomplishments first	
% 	\item 	Your own appraisal of yourself (strengths, weaknesses, uniqueness)
% 		\begin{enumerate}
% 			\item 	Leadership role activities
% 			\item 	Career objectives and goals	
% 		\end{enumerate}
% 	  \hrulefill
% 	  \hrulefill
% \item Personal stories can sometimes be effective, particularly stories of hardships overcome or of an emerging sense of purpose.
% 
% \hrulefill
% 
% 	  \hrulefill
% 	  
% \item  	What makes me different from other grad school applicants?
% 	\item  	Personal and other areas of interest (Hobbies, sports, social or leisure).
% 	\item  	Any other information, which you feel, will support your application
% 	\item Give examples of personal attributes or qualities that would help you complete graduate study successfully. 
% 		  \hrulefill
% 		  
% 	  \hrulefill
% 	  
% \item  Who am I?		
% \begin{itemize}
% 	\item What characteristics do I possess (e.g. honest, compassionate, loyal)?
% 	\item What skills do I have (e.g. analytical, communication, organizational)?
% 	\item How have I changed/grown over the years? What caused these changes and how have they affected me?
% 	\item What makes me unique? How am I different from other applicants?
% 	\item Why should the admissions committee be interested in me?
% 	\item Are there any obstacles that I had to overcome and how have I dealt with these difficulties from my past?
% 	\item Are there any experiences from my past that have affected my life? Can I relate these experiences to my goals?
% 	\item Who has influenced me over the years (e.g. parent, sibling, teacher, or friend) and how have they influenced me?
% 	\item What are my career goals?
% 	\item Why do I want to continue my studies?
% \end{itemize}
%  	  \hrulefill
%  	  
% 		  \hrulefill
% 		  
% 	\end{enumerate}
% 	  \hrulefill
% 	  
% \subsection{Motivations}
% \label{sec:Motivations}
% \begin{enumerate}
% 	\item Explain your motivations.  (of how you decided that teaching is the ideal career path for you)
% 		\item The experiences that made you want to enter this field
% 		\item Steps you have already taken towards these goals   
% 		\item why you feel so passionate about your subject area?  (1.	How did you become interested in this field) 
% 			  \hrulefill
% 			  
% 	  \hrulefill
% 	  
% 		\item	What experiences confirmed that this is what you really want to study?
% 	\item	When did you realize that this wasn't just a casual interest, but what you actually wanted to do with your life?
% 			\item	A summary of your accomplishments first
% 				\item Background information--people and events that influenced your decision
%  	\item 		The reason why you wish to study your chosen subject 
%  	
%  		  \hrulefill
%  		  
% 	  \hrulefill
% 	  
%  	\item Show, dont tell (Use examples to demonstrate your abilities)
% \begin{itemize}
% 	\item Major life events that have changed you
% 	\item Life events that motivate your education
% 	\item People who have influenced you or motivated you 
% 	\item Traits, work habits, and attitudes that will insure your success your goals
% \end{itemize}
%  		  \hrulefill	
%  		   
% 	  \hrulefill
% 	  
% \item  Who am I?	
% \begin{itemize}
% 	\item What characteristics do I possess (e.g. honest, compassionate, loyal)?
% 	\item What skills do I have (e.g. analytical, communication, organizational)?
% 	\item How have I changed/grown over the years? What caused these changes and how have they affected me?
% 	\item What makes me unique? How am I different from other applicants?
% 	\item Why should the admissions committee be interested in me?
% 	\item Are there any obstacles that I had to overcome and how have I dealt with these difficulties from my past?
% 	\item Are there any experiences from my past that have affected my life? Can I relate these experiences to my goals?
% 	\item Who has influenced me over the years (e.g. parent, sibling, teacher, or friend) and how have they influenced me?
% 	\item What are my career goals?
% 	\item Why do I want to continue my studies?
% \end{itemize}
%  	  \hrulefill	 
%  	  
% 	  \hrulefill
% 	  
% \item When did I become fascinated by my field of study?
% \begin{enumerate}
% 	\item Why am I interested in my field of study?
% 	\item What have I learned about my subject of interest?
% 	\item How has my discipline shaped me? What has my field of study taught me about myself?
% 	\item How can I address my academic record
% \end{enumerate}
%  		  \hrulefill	
%  		   
% 	  \hrulefill
% 	  
% 	\end{enumerate}
% 	 \hrulefill
% 	 
% 	 \hrulefill
% 	 
% \subsection{Goals}
% \label{sec:Goals}
% 	\begin{enumerate}		
% \item 	Your own appraisal of yourself (strengths, weaknesses, uniqueness)
% 		\begin{enumerate}		
% 	\item 	Leadership role activities
% 	\item 	  Career objectives and goals
% 	\end{enumerate}
% 		\item 		 	Your ambitions / goals / expectations.
% 	
% \end{enumerate}
% 	  \hrulefill
% 	  
% 	  \hrulefill
% 	  
% 	  \hrulefill
% 	  
% 	  \hrulefill
% 	  
% 	  \hrulefill
% 	  
% \subsection{Educations}
% \label{sec:Educations}
% \begin{enumerate}
%  	\item	Learning experience(s) that serve as a foundation for your choice of career
% \item 	Describe background characteristics that may have placed you at an educational disadvantage (English language learner, family economic history, lack of educational opportunity, disability, etc.).
% 	\item Leave the reader believing that you are prepared for advanced academic work and will be successful in graduate school .
% 	
% \hrulefill
% 
% \hrulefill
% 
% 	\item Do I have any gaps or inconsistencies on my records (transcript and/or exam scores) that I can explain?
%  
%  
%  \begin{enumerate}
%  
% 	\item Are there any awards, recognition, or honors that I have received and that are worth mentioning?
% 	\item How do field experiences enhance my application?
% \end{enumerate}
%  
% 	
%  	\end{enumerate}
%  		  \hrulefill
%  		  
% 	  \hrulefill
% 	  
% 	  \hrulefill
% 	  
%  	
% \subsection{Experience}
% \label{sec:Experience}
% 
%   \begin{enumerate}
% 	\item  Tailor your past experiences to your goals  (2.	
% 	\item What experiences have contributed toward your preparation for further study in this field?)  
% \item What experiences have contributed toward your preparation for further study in this field
% 			\item Steps you have already taken towards these goals   
% \item	Explain not only what you know about your field, but also what you don't know-where is your knowledge particularly strong 
% \item what areas do you still need to learn more about in order to reach your goals?
% 
% 
% 	  \hrulefill
% 	  
% 	  \hrulefill
% 	  
% 	\item  Tailor your graduate school experience to your goals. (3.	What are your future goals?  		\item	A summary of your accomplishments first
% 					\item		Where you picked up first-hand information/experience about the field
% 	\item 		 	Any experience you have to past study related to your chosen subject.
% 	\item 		 	Any employment experience.
% 	\item Describe your determination to achieve your goals, your initiative and ability to develop ideas, and your ability to work independently.
% 		 
% 		  \item  \hrulefill
% 		  
% 	  \hrulefill
% 	  
% 	\item Show, dont tell (Use examples to demonstrate your abilities)
% \begin{itemize}
% 	\item Hobbies 
% 	\item Projects that you've completed
% 	\item Jobs
% 	\item Responsibilities
% 	\item Accomplishments in the personal and scholastic arena
% 	\item Major life events that have changed you
% 	\item Challenges and hurdles you've overcome
%  	\item Traits, work habits, and attitudes that will insure your success your goals
% \end{itemize}
% 	  \hrulefill
% 	  
% 	  \hrulefill
% 	  
% \item What internships and/or jobs have I had in the past?
% 	 \begin{enumerate}
% 	\item What have I learned from my internship and/or job experiences? What skills have I acquired through my internship and/or job?
% 	\item 	What have I learned from my internship and/or job experiences? What skills have I acquired through my internship and/or job?
% 	\item How are my internship and/or job experiences related to my field of interest? Have my internship and/or job experiences prepared me for my future career?
% 	\item Have I been involved in any social services? How has the experience contributed to my growth and how is it related to my goals?
% 	\item What extracurricular activities have I participated in and how do they contribute to my professional goals?
% 	
% 		  \hrulefill
% 		  
% 	  \hrulefill
% 	  
% \end{enumerate}
% \end{enumerate}
% 	  \hrulefill
% 	  
% 	  \hrulefill
% 	  
% 	  \hrulefill
% 	  
% \subsection{Research Interest}
% \label{sec:ResearchInterest}
% 
%   \begin{enumerate}
% 	\item What are your research interests? 
% 	\item  One of the best ways to do this is to discuss, briefly, an idea in your field that turns you on intellectually. It's an effective essay-opener, and it lets you write about something besides yourself for a bit. There are other benefits as well. The idea you choose to talk about, and your comments on it, often tell an admissions committee more about you than your own self-descriptions can.
% Discussing an idea will catch people's attention and give your readers a reprieve from people writing about themselves. That me-me-me  stuff can get irritating after a while, even though it's what the prompt asks for. A discussion of an idea demonstrates your interest in your field, rather than just describing it. Ultimately, that's more convincing.
% 	\end{enumerate}
% 		  \hrulefill
% 		  
% 	  \hrulefill
% 	  
% 	  \hrulefill
% 	  
% \subsection{Program and institute}
% \label{sec:Program}
%   \begin{enumerate}
% 	\item  How are you a  match  for the program to which you are applying? Explain what attracts you most to the institution/program 
% 	\item 		 	The reasons you wish to study in the UK.
% 	\item  	What makes me different from other grad school applicants?
% \item  	What can I contribute to the graduate program?
% 
% 	  \hrulefill
% 	  
% 	  \hrulefill
% 	  
% \item Who will be reading my personal statement?
%  \begin{enumerate}
%  	\item How can I make my essay compelling to the readers?
% 	\item Why am I applying to this program?
% 	\item Why am I applying to this institution?
% 	\item How will attending this graduate school help me grow as an individual and prepare me for my future career?
% 	\item What do I offer the graduate program. Why should a faculty member take me on as a mentee?
% 	\item Considering these issues will help you find a theme for your essay. The next step entails weighing the information to decide what to include.
% \end{enumerate}
% 	  \hrulefill
% 	  
% 	  \hrulefill
% 	  
% \item 	 Finally, construct story that tells readers about who you are and your goals - and how grad school fits.
% 	\end{enumerate}
% 	  \hrulefill
% 	  
% 	  \hrulefill
% 	  
% 	  \hrulefill
% 	  
% %	\item Explain your motivations.  (of how you decided that teaching is the ideal career path for you) \\
% %	The experiences that made you want to enter this field\\
% %	Steps you have already taken towards these goals   \\
% %	why you feel so passionate about your subject area?  (1.	How did you become interested in this field)  \\
% %		What experiences confirmed that this is what you really want to study?
% %%	When did you realize that this wasn't just a casual interest, but what you actually wanted to do with your life?
% %	\item Explain your Goals 
% %	
% %	\item Tailor your past experiences to your goals  (2.	What experiences have contributed toward your preparation for further study in this field?)\\  
% %What experiences have contributed toward your preparation for further study in this field
% %	
% %	Explain not only what you know about your field, but also what you don't know-where is your knowledge particularly strong, and what areas do you still need to learn more about in order to reach your goals? 
% %\subsection{ Notes:  }
% \section {Organization}
% \begin{enumerate}
% 	\item Brief Introduction  or (1) passionate hook;
% 	\item  Education Background  or segu� to your background in the field; 
%  \item (3) specific classes by title and professors you have had (especially if well-known in the field);
% 	\item Employment History  or  related extracurricular activities (especially if they hint at some personal quality you want to convey);
%  \item any publications or other professional accomplishments in the field (perhaps conference presentations or public readings); 
% 	\item Highlight Purpose for Proposed Course of Study  and explanations about problems in your background (if needed); and 
% 	\item Career Goals 
% 	\item why you have chosen this grad school (name one or two professors and what you know of their specific areas or some feature of the program which specifically attracts you).
% 	\item In Conclusion
% \end{enumerate}
% 	  \hrulefill
% 	  
% 	  \hrulefill
% 	  
% 	  \hrulefill
% 	  \newpage
%%%%%%%%%%%%%%%%%%%%%%%%%%%%%%%%%%%%%%%%%%%%%%%%%%%%%%%%%%%%%%%%%%%%%%%%%%%%%%%%%%%%%%%%%%%%%55
% 	  \section{ sample of other work}
% The boundless possibility of trying out and the instant knowledge of the outcome that stimulates one for further analysis of a rationale in question, is what I find most appealing about Computer Science. Keeping up an inquisitive and explorative attitude, I believe, leads to a constant learning process. This approach adds to the already immense potential for innovation that exists in this field.
% As a student in the final year of undergraduate study for a Bachelor of Engineering degree in Computer Science \& Engineering, I look to graduate study to refine my knowledge and skills in my areas of interest. I believe it will also serve to give direction to my goal of a career as a research professional at an academic or commercial, research-oriented organisation. I intend to pursue an MS degree in order to reach that goal.
%  \subsection{ Academic Background \& Research Interests}
%  In my undergraduate studies, I have benefited from the breadth of Shivaji Universitys syllabi content that has given me a comprehensive exposure to the core areas of Computer Science and a strong conceptual understanding of the same. In these three and half years of study, I have strived to maintain an approach of expending independent effort in all my endeavors. Learning by myself and sharing my knowledge with others has been most worthwhile, when comprehending a concept.
%  Over the past two years, I have developed an interest in the areas of Compiler Construction and Information Retrieval. The Compiler Construction lab sessions last semester required the design of a compiler for C++ programs that were restricted to For loops, Switch case statements and simple input and output. The work involved in this, as well as the courses, Formal Systems and Automata and Compiler Construction, have lead me to appreciate the intricacies involved in this field. However, for my MS degree, I remain open to other topics as well.
%   For my B.E project, I have decided to concentrate on my other area of interest viz. Information Retrieval. Im attempting to evolve a technique that, when employed in a search engine, will maximize the comprehensiveness and precision of the query listings. The ongoing work has introduced me to a vast body of pertinent research, such as the University of Washingtons Dynamic Reference Sifting technique for locating an individuals homepage, among others. This has served to sharpen my inclination to engage in active research within this area.
% 	  \subsection{ (Please see attached Resume  for the Academic and Co-curricular distinctions)}
% One persons life influences the lives of an unbelievable number of people, one of the most important lessons I learned, being the Head of the Apex Body at senior secondary level (11th and 12th standard). Hence the need of responsible actions. Becoming the youngest Head of the Apex Body ever and the only one to have served for a period of two years eventually turned out to be a confidence booster. It taught me that the basis for good work is self-reliance and very importantly, time management. Not all of my entire two years were devoid of shortcomings. I learned to accept both criticism and praise with a positive frame of mind. My work involved organising various events, that brought with them the opportunity to work and interact with various kinds of people. This was a distinctively gratifying experience for me, which I feel would stand in good stead in the future, especially in lieu of any teaching experience.
%   In conclusion, I would like to add that the essence of University education lies in the synergetic relationship between the student and his department. I feel that graduate study at your University will be the most logical extension of my academic pursuits and a major step towards achieving my objectives. I would be grateful to you if Im accorded the opportunity to pursue my graduate studies with financial assistance at your institution and am able to justify your faith in me.
%%%%%%%%%%%%%%%%%%%%%%%%%%%%%%%%%%%%%%%%%%%%%%%%%%%%%%%%%%%%%%%%%%%%%%%%%%%%%%%%%%%%%%%%%%%%%55
%   \newpage
% \section{Essay}
% Artificial Intelligence and Image processing was two of the most interesting courses that I took in my last year in undergraduate school. 
% I always thought of a computer as a super intelligent machine, understanding that simple trivial tasks for humans are considered the most challenging problems in computer science was inspiring. The knowledge was gained when I took Artificial Intelligence and Image processing in my last year in undergraduate school. Taking these courses affected my choice for graduation project, it made me think more of a research project that tends to AI and Machine Vision rather than other disciplines. My graduation project (Face Recognition) was chosen to help me pursuit more understanding on the problems of Machine Vision. 
% //further increase my  in  improve my knowledge and  to I choose my graduation project  to solve one of the Machine vision problems. 
% Teaching and work in resala with the orphans helped me to tone my temper and given to improve my weakness as I wanted to present an ideal for the children that they can imetate. It helped me focus on how could I improve my self more to help myself and my comunity. 
%  My research intereset are in field of software engineering in how to use old experiences, projects, bugs, programs to help improve the quality and speed of the software engineering process. using AI, data mining methods is one of the major ideas that are used to help use old requriments in the new software project.  It may not be a silver bullet that solve the software engineering problems but adding new thinking that may help practiceners. 
%  My interset in helping user interact with the computer as humans, we should be able to sketch , write , speak to a computer and be understood. These are trivial things for humans but they are major chalanges in AI. Until now cursive handwriting like arabic and other langauge are still a chanllanging problem. Recognizing sketch is also another field that only recently had atracted researcher. The dificulty is mainly the amount of ambuigus of how different humans write using different style. 
%  I love about reading is gaining more knowldge , I starve for more knowldge more understanding knowing that the more i read the more i new of new areas that i want to know about and i add the my reading list. I remmeber the first book i read about software engineering, \"twenty years : no silver bullet\" it was amazing book and it made me think more of software engneering it made me read more of the topic and get interseted in the current research and new techolgies that solve these problems.   It is entertainig to live with a book and know mroe about other, how they think and how they feel and what motivate others I feel like i am gainig their experience plus my own.   
%  I force my self for improving my self and any one around me like when i was in the company and i research how to improve our current software process. I introduced to the company many software concept that helped in imporvoing the quality of the final product and the quality of the process it self. I gained more experience in managing other and helping them adopt to new knowledge and how to a here to concepts that will provide effective productive products.   I was persistant in requiring and adopting the changes from every one around me. The persistance that helped me to accomplish my goals in both the company final product and my imporivng my self. 
%  I am proud that i am self motivated  , persistant and always move to improve and add knowldge to my self. My hobbies are reading new books about every thing novels, sceince, humman develpment and even new medical advances.  Trying new techology gadget, programs, protoc	%which activities ???????????????????ol and devices. It comes from my facincation with how techology can change our life and how now we are adopted easily to new technology.  I also have a liesure time in swiming, and helping other give me a great satisfactions on how could i help improve not only my self but others around me and even my coumnity.  
%Organizational and analyszing Skills gathered in the project helped to focus my attentions after graduation towards 
 %There is various defintions for Artificall Intelligance, the one that stuck with me is \" \"    
  %I always thought that a computer can easily programmed to do things that are difficult to humans. 
  %developers and 2 quality engineers
  %My resarch graduation porject fostered my interset in research projects and guided me to continue my studies to master degree.   %My work is also  Teaching assistant in  American Univerity in Cairo.    %My graduation project opened my eyes to other posibilities and dreams. I wanted the continous challenge, knowdge that came with research. After my graduation, I    
% organizional and operations. Learning to conduct evaluation and perforamnce measure then reorganize my team to boost and maximize the productivity of our company was a great asset I am proude of gaining.   
%Throught my career I have gathered interesting skills that are in my view valuable for individuals in research teams.   
% \newpage
%\section{Essay Machine Learning}
 
%introduction and motivationss 
`` Computers can be easily programmed to perform things that humans find difficult'', this statement was my assumption for a long time. In my senior year as an undergraduate in computer engineering major, I found out that trivial functions, such as speaking, listening, perception and understanding, are actually the most challenging problems for computers. Being convinced with the importance of having these tasks performed by computers, I built my interest in Artificial Intelligence (AI) and Computer Vision fields. 
 
 
 This interest encouraged me to look for a research problem that combines both fields; AI and computer vision, in my graduation project. Therefore, I chose my project to be ``Face recognition and Identification''. We investigated various face recognition and skin detection methods. Subsequently, we gathered a new skin color dataset which we used to create a skin color model. This model was used to segment image into either skin or background areas. We mapped the face image into Eigen space this is followed by using a classification method to finally identify a person. This project showed me what it meant to work in research and handle the frustration of wrong or unexplained result. On the other hand, the thrill and the continuous motivation to gain more information, ideas help investigating more creative problem solving and thinking techniques. In short, I knew then that I want to continue to work in the scientific research and particularly in the AI field.
  
	Although my dream was to become a scientific researcher, I believed that I need more practical experience before involvement in another academic research. Thus, I worked for four years as a Software Engineer in EngNet Company. During my work I gathered some valuable research skills as I was engaged in various discussions and negotiations with clients and with my own team members. The experience taught me how to resolve conflicts between different opinions and compromise to reach a final decision that will be acceptable by all parties. I was quickly promoted to Team Leader for a software team. I planned, coordinated, and supervised my team to achieve the company goals. I continually needed to evaluate, the organizations and operations of my team to guide them to better engineering practices that will help enhance and maximize the team and in turn the company performance. Learning new skills on how to supervise and evaluate operations of software projects led me to search on how to implement software practices that will enhance the company performance and products.
	
 These experiences along with my research oriented graduation project emphasized that my dream work is working as researcher in an academic institute. Since 2007, I have been affiliated with the 'Signal and Image Processing' research group in the Electronic Research Institute (ERI). The position offered a research environment which I believe is essential for conducting any constructive research. My work in ERI focused on investigating and evaluating the use of Swarm Intelligence in field of Image processing. I also worked on integrating this work with my Master's thesis in Sketch Recognition and Understanding. I worked on evaluating the efficiency of different Particle Swarm Optimization (PSO) algorithms on segmenting users hand drawn sketch. Engaging in research discussions and projects opened my mind to other fields of research other than my own, which was very helpful to provide me with a breath of information that I would definitely help me as a researcher in the long term.    
 	
	I believe that any academic or scientific researcher must acquire some tutoring skills and experience. Thus, prior to my work in EngNet Company, I worked as Teaching Assistant in the October University for Modern Science and Art (MSA), Egypt. Additionally, I have been working as a Teaching and Research assistant in the American University in Cairo (AUC), since Sep., 2008. My responsibilities include preparing tutorials and course materials as well as correcting assignment and projects. I have been assisting in presenting and advising for various courses ranging from Software Engineering and AI to Logic Design and Embedded Systems. Even though my main goal is to become a researcher, teaching is in itself an enriching experience which improved my skills in both professional and personal aspects. 
	
	Besides working as a teaching assistant in AUC, I also work as a research assistant along with Dr. Sherif Abdelazeem. Our research in AUC focuses on Arabic Digit Recognition specifically analyzing and measuring the efficiency of different features and classification algorithms.  The research focuses on investigating and experimenting new domain specific features to improve the accuracy and speed of current Arabic and English Digit Recognition systems. We are currently extending the research to the recognition of Cursive Arabic handwritten words.   	 
	 
	Machine learning is my main field of research interest as mentioned before, mainly the use of machine learning algorithms in image processing in general and specifically in application which aim on improving Human Computer Interaction (HCI). This field has a wide range of problems like handwriting, sketch and gesture recognition. In the last few years, the attention for these applications increased due to the importance of creating easier, faster and reliable HCI which led to introducing of new interfaces like Table-Tops and tangible interfaces. Furthermore, I am also interested in the usage of artificial intelligence in software engineering; particularly ideas like  generating code based on previous similar requirements or using natural language to represent reusable requirements that can be automatically transformed into running programs. My interest in this area is mainly because of my previous experience as a software engineer where I investigated ideas and procedures to improve and speed the development process	
	
	
 	In conclusion, pursuing a PhD degree will help me achieve my goal as a researcher. My choice of studying abroad is based not only on the research experience that I will gain in  studying in another university but also on the experiences I am sure I will gain by living in a diverse community. The University of Calgary is ranked as one of the top universities in Canada which will provide me with the high caliber qualities that I need to widen my vision. I am sure that both my previous and currently growing teaching and research experiences provide me with the necessary background in Computer Science to pursue further studies. I know that research requires persistence, hard work, and self motivation and management skills. I am proud that my professors and supervisors see me as a responsible, reliable and self motivated person. I hope that my experience will enable me to contribute to the on-going research at the University of Calgary.   

 %\end{letter}
\end{document}
