\chapter{Discussion and Conclusion}
\label{sec:DiscussionConclusion}

\section{Conclusion}
\label{sec:ConclusionConclusion}

This research presented a new approach to sketch recognition using PSO. It was noted that the PSO in general improve the accuracy of the final symbol recognition in the system. The use of different preliminary information (speed, time difference and curvature)  helps to improve of the PSO algorithm over the original algorithms \cite{CruveDivisionSwarm,PolygonApproximationPSO}. The tradeoff between accuracy achieved and time complexity must be further investigated to achieve better results.  The system introduced an efficient method to sketch recognition using the PSO. The features computed was a hybrid of a stroke based features which gives more information on the geometrical and structure of the shape tested.  

The system was tested on three different datasets, a simple presentation symbols, electrical symbols and logic design dataset. The number and the complexity of symbols recognized by the system varied from simple to complex but did not affect the final recognition of the system. Test performed confirmed that PSO achieve better performance and optimization than other algorithms. In general \textsl{ALS2} proved to generate the best result on different datasets. Algorithm \textsl{ALS2} achieved better segmentation result that any other algorithm in shapes that are combination of curves and lines as in digital design dataset. Algorithm \textsl{ALS1} achieve high performance in electrical symbols but gives poor result in logic design set. The result is justified because logic design dataset contains symbols that are differentiated by number of curves in OR-gate and XOr gate, therefore representing the symbol as lines will remove important information. 


 On the other hand, the system still has drawback as it lack in handling over traced strokes.  Another drawback is that until now we only applied the system to single symbols.  Full free hand sketch test has not been applied on the system. %sketched 
%drawback of the system is that it takes more time to segment strokes in comparison with other similar algorithms.

\newpage

\section{Future Work}
\label{sec:FutureWork}


%As you can see form the experiments PSO have proved a spuriously than other system. 
%\section{Future Work}
The next step in this research is to complete the clustering algorithm for fully automated sketch recognition. The clustering must be performed without the user explicit involvement.  The currently used algorithm must be modified to accept error reporting from users. The segmentation algorithm may be more powerful if it can segments the stroke into more types of primitives other than the line and curve. Other area of modification can be the features extraction and classifier. Adding more spatial features and structure features will improve classifications. After segmenting the stroke the system can generate a structural graph from segments and there spatial relations. This graph will be matched with the set of known symbol template to make the identification. Those results along with the SVM classifier results will enhance the system as a whole and achieve better recognition rate.  %Each new stroke is checked if it can be a symbol or is a part of already drawn un completed symbol.  
